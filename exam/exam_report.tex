\documentclass[12pt,a4paper]{article}

\usepackage[utf8]{inputenc}
\usepackage[english]{babel}
\usepackage{amsmath}
\usepackage{amsfonts}
\usepackage{amssymb}
\usepackage{stmaryrd}
\usepackage{graphicx}
\usepackage{lmodern}
\usepackage[left=2cm,right=2cm,top=2cm,bottom=2cm]{geometry}

\usepackage{siunitx}
\usepackage{enumitem}
\usepackage[skip=.38\baselineskip]{parskip}
\usepackage[dvipsnames]{xcolor}   % for \textcolor
\usepackage{listings}
\definecolor{mGreen}{rgb}{0,0.4,0.1}
\definecolor{mGray}{rgb}{0.5,0.5,0.5}
\definecolor{mPurple}{rgb}{0.58,0,0.82}
\definecolor{backgroundColour}{rgb}{0.95,0.95,0.92}

\lstdefinelanguage
	[ia64]{Assembler}
	[x86masm]{Assembler}{
	morekeywords={ld4,st4,br.ctop},
	morecomment=[l]{//}
}
\lstset{
	%backgroundcolor=\color{backgroundColour},   
    commentstyle=\color{mGreen},
    keywordstyle=\color{magenta},
    numberstyle=\tiny\color{mGray},
    stringstyle=\color{mPurple},
    %breakatwhitespace=false,                        
    %captionpos=b,
    %keepspaces=true,                 
    numbers=left,                    
    numbersep=5pt,                  
    showspaces=false,                
    showstringspaces=false,
    showtabs=false,                  
    %tabsize=8,	
    language=[ia64]Assembler,    
  	basicstyle=\small\ttfamily,
  	columns=fullflexible,
  	frame=single,
  	breaklines=true,
  	postbreak=\mbox{\textcolor{red}{$\hookrightarrow$}\space},
}

% for multi-figures
\usepackage{subcaption}

% for graphs
%\usepackage{tikz}
%\usepackage{tkz-graph}
%\input{tkz-graph-patch}
%\usepackage{tikz-qtree}
%\usetikzlibrary{calc, arrows, positioning}

% custom commands
\newcommand{\multilinecell}[1]{\begin{tabular}{@{}c@{}}#1\end{tabular}}
\newcommand{\lang}{L\textsubscript{3}~}
\newcommand{\nil}{\textit{nil}}
\newcommand{\cl}[3]{\texttt{\textcolor{OliveGreen}{#1}#2\textcolor{OliveGreen}{#3}}}
\newcommand{\cps}[3]{\texttt{\textcolor{BurntOrange}{#1}#2\textcolor{BurntOrange}{#3}}}
\newcommand{\ts}[2]{#1\textsubscript{#2}}
\newcommand{\code}[1]{\texttt{#1}}
\newcommand{\tran}[3]{\code{\textlbrackdbl #1\textrbrackdbl #2 #3}}
\newcommand{\cv}{\ts{C}{v}}
\newcommand{\ce}{\ts{C}{e}}
\newcommand{\letp}{\ts{let}{p}}
\newcommand{\letc}{\ts{let}{c}}
\newcommand{\letf}{\ts{let}{f}}
\newcommand{\appc}{\ts{app}{c}}
\newcommand{\appf}{\ts{app}{f}}

\author{Olivér Facklam}
\title{\textsc{CS420 Exam Report\\Exceptions in \lang}}


\begin{document}
\maketitle

\begin{quote}
%\centering
\textsc{\large Exam question:}

Assume we wanted to add exceptions to the \lang language, using constructs similar to Java's \texttt{throw} and \texttt{try}/\texttt{catch} statements. To do this, we could add an additional continuation argument to all functions, the exception-handler continuation. A function that wanted to throw an exception would invoke that continuation instead of the return continuation. Explain this idea in more details, in particular how you would translate the \texttt{throw} and \texttt{try}/\texttt{catch} statements. Explain also how the translation of these statements would interact with inlining. For example, what would happen if a function throwing some exception was inlined into one catching it.
\end{quote}


\section*{Introduction}

This report describes how to add exceptions and exception-handling syntax to the \lang language; how these new features will be represented in our CPS/\lang intermediate representation; and finally what the interaction with inlining is.

\section{\lang syntax}

We introduce the following \texttt{throw} syntax: \cl{(throw }{e}{)}. This expression has no value, but raises an exception with value \texttt{e}. If there is no exception-handling code further up in the call stack, the program will ``crash'', \textit{i.e.} it will halt with an exit code of -1 to indicate an error.

In order to allow exception handling, we also introduce the following syntax: 

\begin{center}
\cl{(try }{\ts{b}{t}}{} \cl{catch(}{n}{)} \cl{}{\ts{b}{c}}{)}
\end{center}

This code will start by evaluating the body \code{\ts{b}{t}} of the \code{try} block. In case an exception is raised, the exception value is bound to the literal \code{n} and the body \code{\ts{b}{c}} of the \code{catch} block is executed. The value of the whole expression is either the value of \code{\ts{b}{t}} if there is no exception, or the value of \code{\ts{b}{c}} otherwise. 
If an exception occurs inside \code{\ts{b}{c}} it will be propagated up in the call stack.

\section{CPS/\lang implementation}

Now that this new \lang syntax is defined, we need to describe how the translation to the CPS/\lang intermediate representation is adapted.

In the translation presented in class, expression values are propagated by using a ``translation context''. The translator of a parent node will give such a context to the translator of a sub-expression, which will fill the hole in the context with the value of the sub-expression.

Since now every expression not only has a value, but can also raise an exception, we need a second translation context which allows propagation of the exception values.

The new C\lang to CPS/\lang translator is of the form: \tran{e}{\ts{C}{exception}}{\ts{C}{value}}. It translates the expression \code{e}, giving its value to the value-context \code{\ts{C}{value}}, and giving any exception that is raised to the exception-context \code{\ts{C}{exception}}.

Here is how the current translations have to be adapted:
\begin{description}
\item \tran{a}{\ce}{\cv} with \code{a} an atom = \code{\cv[a]}
\item \tran{\cl{(let }{((\ts{n}{1} \ts{e}{1})(\ts{n}{2} \ts{e}{2})...) e}{)}}{\ce}{\cv} = \\ \tran{\ts{e}{1}}{\ce}{($\lambda$\ts{v}{1} \cps{(\letp~}{((\ts{n}{1} (id \ts{v}{1}))) \tran{\cl{(let }{((\ts{n}{2} \ts{e}{2})...) e}{)}}{\ce}{\cv}}{)})} \footnote{It might be better to create a single continuation containing the exception context and applying it, instead of duplicating the exception context}
\item \tran{\cl{(let }{() e}{)}}{\ce}{\cv} = \tran{e}{\ce}{\cv}
\item \tran{\cl{(letrec }{\cl{(}{(\ts{f}{1} \cl{(fun }{(\ts{n}{11}\ts{n}{12}...)\ts{e}{1}}{)})...}{)} e}{)}}{\ce}{\cv} = \\ \cps{(\letf~}{\cps{(}{(\ts{f}{1} \cps{(fun }{(\underline{c} \underline{d} \ts{n}{11}\ts{n}{12}...) \\\hspace*{7em} \tran{\ts{e}{1}}{($\lambda$\ts{v}{exc1} \cps{(\appc~}{d \ts{v}{exc1}}{)})}{($\lambda$\ts{v}{val1} \cps{(\appc~}{c \ts{v}{val1}}{)})}}{)})\\\hspace*{3.4em}...}{)} \\\tran{e}{\ce}{\cv}}{)}
\item \tran{\cl{(}{e \ts{e}{1}...}{)}}{\ce}{\cv} = \\\cps{(\letc~}{\cps{(}{(\underline{exc} \cps{(cnt }{(\underline{\ts{r}{e}}) \ce[\ts{r}{e}]}{)}) \\\hspace*{3.4em}(\underline{ret} \cps{(cnt }{(\underline{\ts{r}{v}}) \cv[\ts{r}{v}]}{)})}{)}	\\\hspace*{1em}\tran{e}{($\lambda$\ts{v}{exc} \cps{(\appc~}{exc \ts{v}{exc}}{)})}{($\lambda$\ts{v}{val} \tran{\ts{e}{1}}{($\lambda$\ts{v}{exc1} \cps{(\appc~}{exc \ts{v}{exc1}}{)})}{($\lambda$\ts{v}{val1} ... \\\hspace*{3em}\cps{(\appf~}{\ts{v}{val} ret exc \ts{v}{val1}...}{)})})}}{)}
\end{description}

The other translations can be easily adapted in a similar manner. We also introduce the following new translations:
\begin{description}
\item \tran{\cl{(try }{\ts{e}{t}}{} \cl{catch (}{n}{)} \cl{}{\ts{e}{c}}{)}}{\ce}{\cv} = \\
\cps{(\letc~}{\cps{(}{(\underline{c} \cps{(cnt }{(\underline{r}) \cv[r]}{)}) \\\hspace*{3.4em}(\underline{exc} \cps{(cnt }{(n) \tran{\ts{e}{c}}{\ce}{($\lambda$\ts{v}{c} \cps{(\appc~}{c \ts{v}{c}}{)})}}{)})}{)} \\\hspace*{1em}\tran{\ts{e}{t}}{($\lambda$\ts{v}{exc} \cps{(\appc~}{exc \ts{v}{exc}}{)})}{($\lambda$\ts{v}{t} \cps{(\appc~}{c \ts{v}{t}}{)})}}{)}
\item \tran{\cl{(throw }{e}{)}}{\ce}{\cv} = \\\cps{(\letc~}{\cps{(}{(\underline{exc} \cps{(cnt }{(\underline{r}) \ce[r]}{)})}{)} \tran{e}{($\lambda$\ts{v}{e} \cps{(\appc~}{exc \ts{v}{e}}{)})}{($\lambda$v \cps{(\appc~}{exc v}{)})}}{)}
\end{description}

The entire program is translated with the relevant initial contexts, so as to have the program halt with an exit code of 0 in case of success, and an exit code of -1 in case of an error:

\begin{center}
\tran{p}{($\lambda$\ts{v}{exc} \cps{(halt }{-1}{)})}{($\lambda$\ts{v}{val} \cps{(halt }{0}{)})}
\end{center}

\section{Inlining}

Thanks to our intermediate representation being in continuation passing style, the interaction of inlining with exception handling will not cause any issues, just like the interaction between inlining and the return statement goes smoothly.

Consider the following example code:

\cl{(def }{div \cl{(fun }{(a b)\\
\hspace*{7em}\cl{(if }{\cl{(@}{= b 0}{)} \cl{(throw }{0}{)} \cl{(@}{/ a b}{)}}{)}}{)}}{)}\\
\cl{(@}{byte-write \cl{(try }{(div 42 0)}{} \cl{catch }{(e) 0}{)}}{)}

This will get compiled into the following CPS/\lang code:\footnote{I compiled this by hand so I'm not guaranteeing that this is perfectly accurate}

\cps{(\letf~}{\cps{(}{(div \cps{(fun }{(\underline{ret} \underline{exc} a b)\\
\hspace*{8em}\cps{(\letc~}{\cps{(}{(\underline{c} \cps{(cnt }{(r) \cps{(\appc~}{ret r}{)}}{)})\\
\hspace*{11.4em}(\underline{d} \cps{(cnt }{(r) \cps{(\appc~}{exc r}{)}}{)})\\
\hspace*{11.4em}(\underline{ct} \cps{(cnt }{() \cps{(\letc~}{\cps{(}{(\underline{t} \cps{(cnt }{(r) \cps{(\appc~}{d r}{)}}{)})}{)} \cps{(\appc~}{t 0}{)}}{)}}{)})\\
\hspace*{11.4em}(\underline{cf} \cps{(cnt }{() \cps{(\letp~}{\cps{(}{(\underline{n} \cps{(@}{/ a b}{)})}{)} \cps{(\appc~}{c n}{)}}{)}}{)})}{)}\\
\hspace*{10em}\cps{(if (@}{= b 0}{) ct cf)}}{)}}{)})}{)}\\
\hspace*{3em}\cps{(\letc~}{\cps{(}{(\underline{res} \cps{(cnt }{(r) \cps{(\letp~}{\cps{(}{(\underline{n} \cps{(@}{byte-write r}{)})}{)} \cps{(halt }{0}{)}}{)}}{)})\\
\hspace*{6.4em}(\underline{exc} \cps{(cnt }{(e) \cps{(\appc~}{res 0}{)}}{)})}{)}\\
\hspace*{5em}\cps{(\letc~}{\cps{(}{(\underline{fret} \cps{(cnt }{(r) \cps{(\appc~}{res r}{)}}{)})\\
\hspace*{8.4em}(\underline{fexc} \cps{(cnt }{(r) \cps{(\appc~}{exc r}{)}}{)})}{)}\\
\hspace*{7em}\cps{(\appf~}{div fret fexc 42 0}{)}}{)}}{)}}{)}

When inlining, we will get the following result (other optimisations will also kick in of course):

\cps{(\letc~}{\cps{(}{(\underline{res} \cps{(cnt }{(r) \cps{(\letp~}{\cps{(}{(\underline{n} \cps{(@}{byte-write r}{)})}{)} \cps{(halt }{0}{)}}{)}}{)})\\
\hspace*{3.4em}(\underline{exc} \cps{(cnt }{(e) \cps{(\appc~}{res 0}{)}}{)})}{)}\\
\hspace*{2em}\cps{(\letc~}{\cps{(}{(\underline{fret} \cps{(cnt }{(r) \cps{(\appc~}{res r}{)}}{)})\\
\hspace*{5.4em}(\underline{fexc} \cps{(cnt }{(r) \cps{(\appc~}{exc r}{)}}{)})}{)}\\
\hspace*{4em}\cps{(\letc~}{\cps{(}{(\underline{c} \cps{(cnt }{(r) \cps{(\appc~}{fret r}{)}}{)})\\
\hspace*{7.4em}(\underline{d} \cps{(cnt }{(r) \cps{(\appc~}{fexc r}{)}}{)})\\
\hspace*{7.4em}(\underline{ct} \cps{(cnt }{() \cps{(\letc~}{\cps{(}{(\underline{t} \cps{(cnt }{(r) \cps{(\appc~}{d r}{)}}{)})}{)} \cps{(\appc~}{t 0}{)}}{)}}{)})\\
\hspace*{7.4em}(\underline{cf} \cps{(cnt }{() \cps{(\letp~}{\cps{(}{(\underline{n} \cps{(@}{/ 42 0}{)})}{)} \cps{(\appc~}{c n}{)}}{)}}{)})}{)}\\
\hspace*{6em}\cps{(if (@}{= 0 0}{) ct cf)}}{)}
}{)}}{)}

As we can see, the return and exception continuations work perfectly with the inlining. The inlined function body will simply jump to the appropriate piece of code through an \code{\appc} call.

\section*{Conclusion}

In this short report, we have presented 2 new \lang instructions which add the exception handling feature to the language. We have shown how using an additional ``exception-context'' allows us to adapt the existing translation to CPS/\lang and extend it to include the new syntax. Finally we illustrated the way the continuation passing style with the exception continuation perfectly handles function inlining even in the case of exception handling.

\end{document}